\section{Operating Basics}

\subsection{Prepare your AXIOM Beta for use}

\begin{center}
\includegraphics[height=11cm]{images/Getting_Started}\\
\end{center}

1. Use a micro-USB cable to connect the camera's MicroZed development board (USB UART) to a computer. The MicroZed board is the backmost, red PCB. (There is another micro-USB socket on the Power Board, but that is the JTAG Interface.)\\

2. Connect the ethernet port on the MicroZed to an ethernet port on your computer. You might have to use an ethernet adapter on newer, smaller machines which come without a native ethernet port.\\ 

3. Connect the AC adapter to the camera's Power Board. (The power cord plugs into an adapter that connects to the Power Board; to power the camera off at a later point, you need not disconnect the adapter from the board but can just unplug the cord from the adapter.)


\begin{center}
\includegraphics[height=17cm]{images/BetaGuide}\\
\end{center}

\subsection{Prepare your computer for use with AXIOM Beta}

\textbf{Overview -} To communicate with your AXIOM Beta camera, you will send it instructions via your computer's command line.\\

In case you have not worked with a shell (console, terminal) much or ever before, we have prepared detailed instructions to help you get you set up. The steps which need to 	be taken to prepare your machine sometimes differ between operating systems, so pick the ones that are applicable to you(r system). \\

\textbf{Note:} Dollar signs placed in front of commands are not meant to be typed in but denote the command line prompt (a signal indicating the computer is ready for user input). It is used in documentation to differentiate between commands and output resulting from commands. The prompt might look different on your machine (e.g. an angled bracket >) and be preceded by your user name, computer name or the name of the directory which you are currently inside.



\subsubsection{USB to UART Drivers}

For the USB connection to work, you will need drivers for bridging USB to UART (USB to serial). (Under Linux this works out of the box in most 	distributions) for other operating systems they can be \href{https://www.silabs.com/products/development-tools/software/usb-to-uart-bridge-vcp-drivers}{downloaded} from e.g. Silicon Labs' website – pick the software provided for your OS and install it. \\

\subsubsection{Serial Console}

The tool we recommend for connecting to the AXIOM Beta camera via serial port with Mac OS X or Linux is \href{https://linux.die.net/man/1/minicom}{minicom}; for connections from Windows machines, we have used \href{http://www.putty.org/}{Putty}.

\paragraph{Linux Setup}

Check if you already have minicom installed on your system by trying to run it:

\begin{lstlisting}[language=bash,morekeywords=$,keywordstyle=\bfseries,frame=none,xleftmargin=.25in,belowskip=2em, aboveskip=2em]
$ minicom
\end{lstlisting}

Your system will respond with a message like \importantKeyword{bash: command not found: minicom} if it's not installed.\\

\textbf{Install minicom}\\

Install the minicom package like you'd install other software on your system – which could be via a GUI tool or using \importantKeyword{aptitude} or \importantKeyword{apt-get} (for wich you might need super user rights), e.g.: 

\begin{lstlisting}[language=bash,morekeywords=$,keywordstyle=\bfseries,frame=none,xleftmargin=.25in,belowskip=2em, aboveskip=2em]
$ apt-get install minicom
\end{lstlisting}

or

\begin{lstlisting}[language=bash,morekeywords=$,keywordstyle=\bfseries,frame=none,xleftmargin=.25in,belowskip=2em, aboveskip=2em]
$ sudo apt-get install minicom
\end{lstlisting}

\paragraph{MAC OSX Setup}

You will want to have \href{https://brew.sh/}{Homebrew} installed on your system to use \importantKeyword{minicom} for serial communication as it is more convenient than using \importantKeyword{screen}.\\

\textbf{Note:} Homebrew is a package manager for Mac, a piece of software that helps you install other software on your Mac machine, particularly software which is readily available on Linux but which does not come in the form of Mac "applications", which you can download via your web browser and simply drop into your Applications folder.\\ 

Open Terminal.app (or your preferred terminal emulator if you have another installed). Terminal can be found via e.g. Spotlight search or via the Finder menu: \importantKeyword{Go > Utilities > Terminal.app}.\\

Check if you already have brew installed by entering the brew command: 

\begin{lstlisting}[language=bash,morekeywords=$,keywordstyle=\bfseries,frame=none,xleftmargin=.25in,belowskip=2em, aboveskip=2em]
$ brew
\end{lstlisting}

If you don't have Homebrew installed, your shell will reply with something like \importantKeyword{bash: command not found: brew}. Otherwise, it will spit out a list of \importantKeyword{brew} commands.\\ 

\textbf{Install Homebrew}\\

To install Homebrew, go to the \href{https://brew.sh/}{Homebrew} website and follow the install instructions there. You can simply copy the command used for installing Homebrew from their website and paste it into your terminal.\\

\textbf{Install minicom}\\

With brew installed, you want to install minicom:

\begin{lstlisting}[language=bash,morekeywords=$,keywordstyle=\bfseries,frame=none,xleftmargin=.25in,belowskip=2em, aboveskip=2em]
$ brew install minicom
\end{lstlisting}

Homebrew will tell you if you already have minicom installed on your system (e.g. \importantKeyword{Warning: minicom-2.7 already installed}), otherwise it will install it for you. 


\paragraph{Minicom Configuration}

Once you have minicom installed, you need to configure it in order to talk to the camera.\\

You can either use the configuration file we prepared or configure it yourself, following our step-by-step instructions. \\

\textbf{Use Settings File}\\

\textbf{Linux}\\

Go to the minicom setup page:

\begin{lstlisting}[language=bash,morekeywords=$,keywordstyle=\bfseries,frame=none,xleftmargin=.25in,belowskip=2em, aboveskip=2em]
$ minicom -s
\end{lstlisting}

In the "Serial port setup" subpage, check that "Serial Device" 's name is the good one (usually /dev/ttyUSB0 on Linux) and check the baud rate (115200).\\

\textbf{Mac}\\

Download the \href{https://wiki.apertus.org/images/0/06/Minirc.USB0_Mac.zip}{Settings File} for Mac, unzip it and place it in the \importantKeyword{etc} directory of your minicom install.

The minicom installation can be found in the standard directory used by homebrew, \importantKeyword{/usr/local/Cellar}, in a subdirectory based on the minicom version number, e.g. \importantKeyword{/usr/local/Cellar/minicom/2.7/etc}.\\

You can also use Homebrew's \importantKeyword{info} command to find minicom on your hard disk: 

\begin{lstlisting}[language=bash,morekeywords=$,keywordstyle=\bfseries,frame=none,xleftmargin=.25in,belowskip=2em, aboveskip=2em]
$ brew info minicom
\end{lstlisting}

... which will output general information on the installed package, including its install directory e.g.: 

\begin{lstlisting}[breaklines=true, breakatwhitespace=true]
    minicom: stable 2.7 (bottled)
    Menu-driven communications program
    https://alioth.debian.org/projects/minicom/
    /usr/local/Cellar/minicom/2.7 (17 files, 346.6K) *
      Poured from bottle on ...
\end{lstlisting}

\subsubsection{Serial connection (via USB)}

While the AXIOM Beta can be connected to via USB UART (USB to serial), a serial connection is not the preferred way to communicate with the camera but rather in place for monitoring purposes.\\

However, a serial connection is needed to set up communication via ethernet/LAN (the better suited way to talk to the camera): as the Beta only allows for secure ethernet connections, you will have to connect to the camera via serial port first and copy over your SSH key.\\

Below are two different methods for connecting to the AXIOM Beta camera, using a program called minicom and an alternative program called screen. We suggest you try them in the order below, so if you can connect with minicom great, if not try screen. \\


\paragraph{Connect using Minicom}\mbox{}\\

\textbf{Note :} You will not be able to use the terminal window you initiate the serial connection in for anything else (it needs to remain open while you access the camera), so it might make sense to open a separate window just for this purpose.\\

With minicom installed and properly configured, all you need to do is run the following command to start it with the correct settings:

\begin{lstlisting}[language=bash,morekeywords=$,keywordstyle=\bfseries,frame=none,xleftmargin=.25in,belowskip=2em, aboveskip=2em]
$ minicom -8 USB0
\end{lstlisting}

On successful connection, you will be prompted to enter user credentials (which are needed to log into the camera).\\

If your terminal remains blank except for the minicom welcome screen/information about your connection settings, try pressing enter. If this still does not result in the prompt for user credentials – while testing, we discovered the initial connection with minicom does not always work – disconnect the camera from the power adapter, then reconnect it: in your minicom window you should now see the camera's operating system booting up, followed by the login prompt. (From then on, connecting with minicom should work smoothly and at most require you to press enter to make the login prompt appear.)\\

The default credentials are: 

\begin{lstlisting}[language=bash,morekeywords=$,keywordstyle=\bfseries,frame=none,xleftmargin=.25in,belowskip=2em, aboveskip=2em]
    user: root
    password: beta
\end{lstlisting}

\textbf{Alternative tools for serial connection}\\

Before you can use any tool to initiate a serial connection with your Beta camera, you need to know through which special device file it can be accessed.\\

Once the Beta is connected and powered on (and you installed the necessary drivers), it gets listed as a USB device in the \importantKeyword{/dev} directory of your file system, e.g.\\

\importantKeyword{/dev/ttyUSB0} (on Linux)

or

\importantKeyword{/dev/cu.SLAB\_USBtoUART}\\

\importantKeyword{/dev/tty.SLAB\_USBtoUART} (on Mac).

You can use a command such as: 

\begin{lstlisting}[language=bash,morekeywords=$,keywordstyle=\bfseries,frame=none,xleftmargin=.25in,belowskip=2em, aboveskip=2em]
$ ls -al /dev | grep -i usb
\end{lstlisting}

... to list all USB devices currently connected to your machine. 



\paragraph{Connect using Screen}\mbox{}\\
To connect to the camera, use the command: 

\begin{lstlisting}[language=bash,morekeywords=$,keywordstyle=\bfseries,frame=none,xleftmargin=.25in,belowskip=2em, aboveskip=2em]
$ screen file_path 115200
\end{lstlisting}

... where \importantKeyword{file\_path} is the full path to the special device file (e.g. \importantKeyword{/dev/ttyUSB0} or \importantKeyword{/dev/cu.SLAB\_USBtoUART}).

You might have to run the command with superuser rights, i.e.: 

\begin{lstlisting}[language=bash,morekeywords=$,keywordstyle=\bfseries,frame=none,xleftmargin=.25in,belowskip=2em, aboveskip=2em]
$ sudo screen file_path 115200
\end{lstlisting}

On successful connection, you will be prompted to enter user credentials needed for logging into the camera.
If your terminal remains blank, try pressing enter.\\

The default credentials are: 

\begin{lstlisting}[language=bash,morekeywords=$,keywordstyle=\bfseries,frame=none,xleftmargin=.25in,belowskip=2em, aboveskip=2em]
    user: root
    password: beta
\end{lstlisting}


\paragraph{Disconnect}\mbox{}\\

To exit the camera's operating system, use: 

\begin{lstlisting}[language=bash,morekeywords=$,keywordstyle=\bfseries,frame=none,xleftmargin=.25in,belowskip=2em, aboveskip=2em]
$ exit
\end{lstlisting}

The result will be a logout message followed by a new login prompt.

To suspend or quit your \importantKeyword{screen} session (and return to your regular terminal window) use one of the following commands: 

\begin{lstlisting}[language=bash,morekeywords=$,keywordstyle=\bfseries,frame=none,xleftmargin=.25in,belowskip=2em, aboveskip=2em]
CTRL+a CTRL+z
\end{lstlisting}

\begin{lstlisting}[language=bash,morekeywords=$,keywordstyle=\bfseries,frame=none,xleftmargin=.25in,belowskip=2em, aboveskip=2em]
CTRL+a CTRL+\
\end{lstlisting}


\subsubsection{Ethernet connection (using SSH)}
\paragraph{SSH Keys how-to for Linux and Mac}
\subparagraph{Storage location/Find existing keys}
\subparagraph{SSH key creation}
\paragraph{Get or set IP address}
\paragraph{IP address check}
\paragraph{Set IP address}
\paragraph{Establish a connection via key}
\paragraph{Password-based authentication}
\subsubsection{Start the camera}
\subsubsection{WiFi access point setup}

\subsubsection{WiFi access point setup}


A neat tool for communicating with your Beta camera via serial port is \importantKeyword{minicom}.

Check if you already have minicom installed on your system by trying to run it:

\begin{lstlisting}[language=bash,morekeywords=$,keywordstyle=\bfseries,frame=none,xleftmargin=.25in,belowskip=2em, aboveskip=2em]
$ minicom
\end{lstlisting}
Your system will respond with a message like bash: command not found: minicom if it's not installed.


\begin{lstlisting}[language=bash,morekeywords=$,keywordstyle=\bfseries,frame=none,xleftmargin=.25in,belowskip=2em, aboveskip=2em]
$ minicom
\end{lstlisting}

Your system will respond with a message like \importantKeyword{bash: command not found: minicom} if it's not installed.\\

\textbf{Install minicom}\\

Install the minicom package like you'd install other software on your system – which could be via a GUI tool or using \importantKeyword{aptitude} or \importantKeyword{apt-get} (for wich you might need super user rights), e.g.: 

\begin{lstlisting}[language=bash,morekeywords=$,keywordstyle=\bfseries,frame=none,xleftmargin=.25in,belowskip=2em, aboveskip=2em]
$ apt-get install minicom
\end{lstlisting}

or

\begin{lstlisting}[language=bash,morekeywords=$,keywordstyle=\bfseries,frame=none,xleftmargin=.25in,belowskip=2em, aboveskip=2em]
$ sudo apt-get install minicom
\end{lstlisting}

\paragraph{MAC OSX Setup}

You will want to have \href{https://brew.sh/}{Homebrew} installed on your system to use \importantKeyword{minicom} for serial communication as it is more convenient than using \importantKeyword{screen}.\\

\textbf{Note:} Homebrew is a package manager for Mac, a piece of software that helps you install other software on your Mac machine, particularly software which is readily available on Linux but which does not come in the form of Mac "applications", which you can download via your web browser and simply drop into your Applications folder.\\ 

Open Terminal.app (or your preferred terminal emulator if you have another installed). Terminal can be found via e.g. Spotlight search or via the Finder menu: \importantKeyword{Go > Utilities > Terminal.app}.\\

Check if you already have brew installed by entering the brew command: 

\begin{lstlisting}[language=bash,morekeywords=$,keywordstyle=\bfseries,frame=none,xleftmargin=.25in,belowskip=2em, aboveskip=2em]
$ brew
\end{lstlisting}

If you don't have Homebrew installed, your shell will reply with something like \importantKeyword{bash: command not found: brew}. Otherwise, it will spit out a list of \importantKeyword{brew} commands.\\ 

\textbf{Install Homebrew}\\

To install Homebrew, go to the \href{https://brew.sh/}{Homebrew} website and follow the install instructions there. You can simply copy the command used for installing Homebrew from their website and paste it into your terminal.\\

\textbf{Install minicom}\\

With brew installed, you want to install minicom:

\begin{lstlisting}[language=bash,morekeywords=$,keywordstyle=\bfseries,frame=none,xleftmargin=.25in,belowskip=2em, aboveskip=2em]
$ brew install minicom
\end{lstlisting}

Homebrew will tell you if you already have minicom installed on your system (e.g. \importantKeyword{Warning: minicom-2.7 already installed}), otherwise it will install it for you. 


\paragraph{Minicom Configuration}
\subsubsection{Serial connection (via USB)}
\paragraph{Connect using Minicom}\mbox{}\\
Note that you will not be able to use the terminal window you initiate the serial connection in for anything else (it needs to remain open while you access the camera), so it might make sense to open a separate window just for this purpose.

With minicom installed and properly configured, all you need to do is run the following command to start it with the correct settings:

\begin{lstlisting}[language=bash,morekeywords=$,keywordstyle=\bfseries,frame=none,xleftmargin=.25in,belowskip=2em, aboveskip=2em]
$ minicom -8 USB0
\end{lstlisting}

On successful connection, you will be prompted to enter user credentials (which are needed to log into the camera).\\

Test\\

If your terminal remains blank except for the minicom welcome screen/information about your connection settings, try pressing enter. If this still does not result in the prompt for user credentials – while testing, we discovered the initial connection with minicom does not always work – disconnect the camera from the power adapter, then reconnect it: in your minicom window you should now see the camera's operating system booting up, followed by the login prompt. (From then on, connecting with minicom should work smoothly and at most require you to press enter to make the login prompt appear.)

\paragraph{Connect using Screen}
\paragraph{Disconnect}
\subsubsection{Ethernet connection (using SSH)}
\paragraph{SSH Keys how-to for Linux and Mac}
\subparagraph{Storage location/Find existing keys}
\subparagraph{SSH key creation}
\paragraph{Get or set IP address}
\paragraph{IP address check}
\paragraph{Set IP address}
\paragraph{Establish a connection via key}
\paragraph{Password-based authentication}
\subsubsection{Start the camera}
\subsubsection{WiFi access point setup}

\subsubsection{WiFi access point setup}


A neat tool for communicating with your Beta camera via serial port is \importantKeyword{minicom}.

Check if you already have minicom installed on your system by trying to run it:

\begin{lstlisting}[language=bash,morekeywords=$,keywordstyle=\bfseries,frame=none,xleftmargin=.25in,belowskip=2em, aboveskip=2em]
$ minicom
\end{lstlisting}
Your system will respond with a message like bash: command not found: minicom if it's not installed.

\begin{lstlisting}[language=bash,morekeywords=$,keywordstyle=\bfseries,frame=none,xleftmargin=.25in,belowskip=2em, aboveskip=2em]
$ minicom
\end{lstlisting}

Your system will respond with a message like bash: command not found: minicom if it's not installed.





\paragraph{MAC OSX Setup}
\paragraph{Minicom Configuration}
\subsubsection{Serial connection (via USB)}
\paragraph{Connect using Minicom}\mbox{}\\
Note that you will not be able to use the terminal window you initiate the serial connection in for anything else (it needs to remain open while you access the camera), so it might make sense to open a separate window just for this purpose.

With minicom installed and properly configured, all you need to do is run the following command to start it with the correct settings:

\begin{lstlisting}[language=bash,morekeywords=$,keywordstyle=\bfseries,frame=none,xleftmargin=.25in,belowskip=2em, aboveskip=2em]
$ minicom -8 USB0
\end{lstlisting}

On successful connection, you will be prompted to enter user credentials (which are needed to log into the camera).\\

Test\\

If your terminal remains blank except for the minicom welcome screen/information about your connection settings, try pressing enter. If this still does not result in the prompt for user credentials – while testing, we discovered the initial connection with minicom does not always work – disconnect the camera from the power adapter, then reconnect it: in your minicom window you should now see the camera's operating system booting up, followed by the login prompt. (From then on, connecting with minicom should work smoothly and at most require you to press enter to make the login prompt appear.)

\paragraph{Connect using Screen}
\paragraph{Disconnect}
\subsubsection{Ethernet connection (using SSH)}
\paragraph{SSH Keys how-to for Linux and Mac}
\subparagraph{Storage location/Find existing keys}
\subparagraph{SSH key creation}
\paragraph{Get or set IP address}
\paragraph{IP address check}
\paragraph{Set IP address}
\paragraph{Establish a connection via key}
\paragraph{Password-based authentication}
\subsubsection{Start the camera}
\subsubsection{WiFi access point setup}

\subsubsection{WiFi access point setup}

A neat tool for communicating with your Beta camera via serial port is \importantKeyword{minicom}.

Check if you already have minicom installed on your system by trying to run it:

\begin{lstlisting}[language=bash,morekeywords=$,keywordstyle=\bfseries,frame=none,xleftmargin=.25in,belowskip=2em, aboveskip=2em]
$ minicom
\end{lstlisting}
Your system will respond with a message like bash: command not found: minicom if it's not installed.
